\documentclass[
	lecture={4},
	title={Introduction to Probability}
]{msf502notes}

\begin{document}

\maketitle

\setcounter{chapter}{3}
%<*Chapter-4>
\chapter{Introduction to Probability}\label{ch:introduction-to-probability}
\section{Sportsware Brands}\label{sec:sportsware-brands}
\begin{itemize}
	\item Annabel Gonzalez, chief retail analyst at marketing firm Longmeadow Consultants is tracking the sales of compression-gear produced by Under Armour, Inc., Nike, Inc, and Adidas Group.
	\item After collecting data from 600 recent purchases, Annabel wants to determine whether age influences brand choice.
\end{itemize}


\begin{table}[H]
    \centering
    \begin{tabular}{|l|c|c|c|}
        \toprule
        \textbf{} & \multicolumn{3}{|c|}{\textbf{Brand Name}}\\
        \midrule
        \textbf{Age Group} & \textbf{Under Armour} & \textbf{Nike} & \textbf{Adidas} \\
		\midrule
		\textbf{Under 25 years} & 174 & 132 & 90 \\
		\midrule
		\textbf{35 years and older} & 54 & 72 & 78 \\
        \bottomrule
    \end{tabular}
    \caption{Live Example for Chapter 4}
    \label{tab:chapter-4}
\end{table}

\section{Fundamental Probability Concepts}\label{sec:fundamental-probability-concepts}
\begin{itemize}
	\item A probability is a numerical value that $\dots$
	\item An experiment
\end{itemize}

% TODO: Went through a lot of slides very fast

\subsection{Assigning Probabilities}\label{subsec:assigning-probabilities}
\subsubsection{Subjective Probabilities}\label{subsubsec:subjective-probabilities}
\begin{itemize}
	\item Draws on personal and subjective judgment.
\end{itemize}

\subsubsection{Objective Probabilities}\label{subsubsec:objective-probabilities}
\begin{itemize}
	\item Empirical probability: a relative frequency of occurrence
	\item a priori probability: a logical analysis
\end{itemize}

\subsection{Probabilities expressed as odds}\label{subsec:probabilities-expressed-as-odds}
\emph{Percentages} and \emph{odds} are an alternative approach to expressing probabilities include.

\subsection{Converting an odds ratio to a probability}\label{subsec:converting-an-odds-ratio-to-a-probability}
Given \emph{odds} for event A occurring of ``$a$ to $b$'', the probability of $A$ is:
\begin{equation}
	\frac{a}{a+b}
	\label{eq:odds-for}
\end{equation}

Given \emph{odds} against event A occurring of ``$a$ to $b$'', the probability of $A$ is:
\begin{equation}
	\frac{b}{a+b}
	\label{eq:odds-against}
\end{equation}

\subsection{Converting probability to an odds ratio}\label{subsec:converting-probability-to-an-odds-ratio}

\section{Rules of Probability}\label{sec:rules-of-probability}
\begin{equation}
	P(A | B) = \frac{P(A \cap B)}{P(B)}
	\label{eq:conditional-probability}
\end{equation}

\subsection{Multiplication Rule}\label{subsec:multiplication-rule}
\begin{equation}
	P(A \cap B) = P(A | B) \times P(B) = P(B | A) \times P(A)
	\label{eq:multiplication-rule}
\end{equation}

\section{Contingency Tables and Probabilities}\label{sec:contingency-tables-and-probabilities}
\subsection{Contingency Tables}\label{subsec:contingency-tables}
\begin{itemize}
	\item A contingency table generally shows frequencies for two qualitative $\dots$
\end{itemize}

\section{Bayes' Rule}\label{sec:bayes'-rule}
\begin{equation}
	\begin{aligned}
		P(B | A) &= \frac{P(A \cap B)}{P(A \cap B) + P(A \cap B^{c})}\\
		&= \frac{P(A | B)P(B)}{P(A | B)P(B) + P(A | B^{c})P(B)^{c}}\\
	\end{aligned}
	\label{eq:Bayes-Theorem}
\end{equation}


\begin{table}
    \centering
    \begin{tabular}{llll}
        \toprule
        \textbf{Prior Probability} & \textbf{Conditional Probability} & \textbf{Joint Probability} & \textbf{Posterior Probability} \\
        \midrule
        $P(T)$ = 0.99 & & & \\
		\midrule
		$P(T^{c})$ = 0.01 & & & \\
		\midrule
		$P(T) + P(T^{c})$ = 1 & & & \\
        \bottomrule
    \end{tabular}
    \caption{Bayes' Rule Example}
    \label{tab:bayes-rule-example}
\end{table}

We find:
\[ \begin{aligned}
	P(T | D) &= \frac{(0.005)(0.99)}{(0.005)(0.99) + (0.95)(0.01)}\\
	&= \frac{0.00495}{0.00495 + 0.0095}\\
	&= \frac{0.00495}{0.01445}\\
	&= 0.342560554\\
\end{aligned} \]

\section{Counting Rules}\label{sec:counting-rules}
\begin{equation}
	_{n}C_{x} = \left( \begin{array}{c}
		n\\ x
	\end{array} \right) = \frac{n!}{(n-x)!x!}
	\label{eq:combination-rule}
\end{equation}

\begin{equation}
	_{n}P_{x} = \dots
	\label{eq:permutation-rule}
\end{equation}
%</Chapter-4>


\end{document}