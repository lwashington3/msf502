\documentclass[
	lecture={2},
	title={}
]{msf502notes}

\begin{document}

\maketitle

\setcounter{chapter}{1}
%<*Chapter-2>
\chapter{Chapter 2 Name Here}
\section{Tabular $\dots$}\label{sec:tabular-...}

\section{Summarizing Qualitative Data}\label{sec:summarizing-qualitative-data}

\begin{itemize}
    \item A bar chart depicts the frequency or the relative frequency for each category of the qualitative data as a bar rising vertically from the horizontal axis.
	\item For example, Adidas' sales may be proportionally compared for each Region over these two periods.
\end{itemize}

% Insert bar chart figure

\begin{itemize}
	\item A frequency distribution for quantitative data groups data into intervals called classes, and records the number of observations that fall into each class.
	\item Guidelines when constructing frequency distribution:
	\begin{itemize}
		\item Classes are \textit{mutually exclusive}.
		\item Classes are \textit{exhaustive}.
	\end{itemize}
	\item The number of classes usually ranges from 5 to 20.
	\item Approximating the class width:
	\[ \frac{\text{Largest Value - Smallest Value}}{\text{\# of classes}} \]
	\item The
	\item A cumulative frequency distribution specifies how many observations fall below the upper limit of a particular class.
\end{itemize}
%</Chapter-2>

\end{document}
