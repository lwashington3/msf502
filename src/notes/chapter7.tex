\documentclass[
	lecture={7}, % Lecture 6 was the first exam
	title={Sampling and Sampling Distributions}
]{msf502notes}

\begin{document}

\setcounter{chapter}{6}
%<*Chapter-7>
\chapter{Sampling and Sampling Distributions}\label{ch:sampling-and-sampling-distributions}
\begin{objectives}
	\item Differentiate between a population parameter and a sample statistic.
	\item Explain common sample biases.
	\item Describe simple random sampling.
	\item Distinguish between stratified random sampling and cluster sampling.
	\item Describe the properties of the sampling distribution of the sample mean.
	\item Explain the importance of the central limit theorem.
	\item Describe the properties of the sample distribution of the sample proportion.
	\item Use a finite population correction factor.
	\item Construct and interpret control charts from quantitative and qualitative data.
\end{objectives}

%Population parameters such as population mean, population standard deviation, population proportion.
%For each population parameter, there is a matching sample statistic: sample mean, sample standard deviation, sample proportion.
%
%The objective is to learn about the population mean:

\section{Sampling}\label{sec:sampling}
\lo{Differentiate between a population parameter and a sample statistic.}
\begin{itemize}
	\item \definition{Population}{consists of all items of interest in a statistical problem.}
	\begin{itemize}
		\item \emph{Population Parameter} is unknown.
	\end{itemize}
	\item \definition{Sample}{a subset of the population.}
	\begin{itemize}
		\item \emph{Sample statistic} is calculated from sample and used to make inferences about the population.
	\end{itemize}
	\item \definition{Bias}{the tendency of a sample statistic to systematically over- or under-estimate a population parameter.}
\end{itemize}

\lo{Explain common sample biases.}
\begin{itemize}
	\item Classic Case of a ``Bad'' Sample: The \textit{Literary Digest} Debacle of 1936
	\begin{itemize}
		\item During the 1936 presidential election, the \textit{Literary Digest} predicted a landslide victory of Alf Landon over Franklin D.\ Roosevelt (FDR) \w\ only a 1\% margin or error.
		\item They were wrong!
		FDR won in a landslide election.
		\item The \textit{Literary Digest} had committed \emph{selection bias} by randomly sampling from their own subscriber/membership lists, etc.
		\item In addition, \w\ only a 24\% response rate, the \textit{Literary Digest} had a great deal of non-response bias.
	\end{itemize}
\end{itemize}

\begin{itemize}
	\item \definition{Selection bias}{a systematic exclusion of certain groups from consideration for the sample.}
	\begin{itemize}
		\item The \textit{Literary Digest} committed selection bias by excluding a large portion of the population (e.g., lower income voters).
	\end{itemize}
	\item \definition{Nonresponse bias}{a systematic difference in preferences between respondents and non-respondents to a survey or a poll.}
	\begin{itemize}
		\item The \textit{Literary Digest} had only a 24\% response rate.
		This indicates that only those who cared a great deal about the election took the time to respond to the survey.
		These respondents may be atypical of the population as a whole.
	\end{itemize}
\end{itemize}

\lo{Describe simple random sampling.}
\subsection{Sampling Methods}\label{subsec:sampling-methods}
\begin{itemize}
	\item Simple random sample is a sample of $n$ observations which have the sample probability of being selected from the population as any other sample of $n$ observations.
	\begin{itemize}
		\item Most statistical methods presume simple random samples.
		\item However, in some situations, other sampling methods have an advantage over simple random samples.
	\end{itemize}
\end{itemize}

\lo{Distinguish between stratified random sampling and cluster sampling.}
\subsection{Stratified Random Sampling}\label{subsec:stratified-random-sampling}
\begin{itemize}
	\item Divide the population into mutually exclusive and collectively exhaustive groups, called \emph{strata}.
	\item Randomly select observations from each stratum, which are proportional to the stratum's size.
	\item Advantages:
	\begin{itemize}
		\item Guarantees that each population's subdivision is represented in the sample.
		\item Parameter estimates have greater precision than those estimated from simple random sampling.
	\end{itemize}
\end{itemize}

\subsection{Cluster Sampling}\label{subsec:cluster-sampling}
\begin{itemize}
	\item Divide population into mutually exclusive and collectively exhaustive groups, called clusters.
	\item Random select clusters.
	\item Sample every observation in those randomly selected clusters.
	\item Advantages and disadvantages:
	\begin{itemize}
		\item Less expensive than other sampling methods.
		\item Less precision than simple random sampling or stratified sapling.
		\item Useful when clusters occur naturally in the population.
	\end{itemize}
\end{itemize}

\begin{table}[H]
	\centering
	\caption{Stratified vs.\ Cluster Sampling}
	\label{tab:stratified-vs-cluster-sampling}
	\begin{tabular}{|p{0.475\textwidth}|p{0.475\textwidth}|}
		\hline
		\textbf{Stratified Sampling} & \textbf{Cluster Sampling}\\
		\hline
		Sample consists of elements from each group. & Sample consists of elements from the selected groups.\\
		\hline
		Preferred when the objective is to increase precision. & Preferred when the objective is to reduce costs.\\
		\hline
	\end{tabular}
\end{table}

\section{The Sampling Distribution of the Means}\label{sec:the-sampling-distribution-of-the-means}
\lo{Describe the properties of the sampling distribution of the same mean.}
\begin{itemize}
	\item Population is described by parameters.
	\begin{itemize}
		\item A \textit{parameter} is a constant, whose value may be unknown.
		\item Only one population.
	\end{itemize}
	\item Sample is described by statistics.
	\begin{itemize}
		\item A \emph{statistic} is a random variable whose value depends on the chosen random sample.
		\item Statistics are used to make \emph{inferences} about the population parameters.
		\item Can draw multiple random samples of size $n$.
	\end{itemize}
\end{itemize}

\subsection{Estimator}\label{subsec:sampling-distributor-estimator}
\begin{itemize}
	\item A statistic that is used to estimate a population parameter.
	\item For example, $\bar{X}$, the mean of the sample, is an estimate of $\mu$, the mean of the population.
\end{itemize}

\subsection{Estimate}\label{subsec:sampling-distributor-estimate}
\begin{itemize}
	\item A particular value of the estimator.
	\item For example, the mean of the sample $\bar{x}$ is an estimate of $\mu$, the mean of the population.
\end{itemize}

\subsection{Sampling Distribution of the Mean $\bar{x}$}\label{subsec:sampling-distribution-of-the-mean-x}
\begin{itemize}
	\item Each random sample size $n$ drawn from the population provides an estimate of $\mu$--the sample mean $\bar{x}$.
	\item Drawing many samples of size $n$ results in many different sample means, one for each sample.
	\item The sampling distribution of the mean is the frequency or probability distribution of these sample means.
\end{itemize}

\subsection[E.V. and S.D. of the Sample Mean]{The Expected Value and Standard Deviation of the Sample Mean}\label{subsec:the-expected-value-and-standard-deviation-of-the-sample-mean}
\begin{itemize}
	\item The expected value of $X$,
	\begin{equation}
		E(X) = \mu
		\label{eq:expected-value-random-variable}
	\end{equation}
	\item The expected value of the mean,
	\begin{equation}
		E(\bar{X}) = E(X) = \mu
		\label{eq:expected-value-sample-mean}
	\end{equation}
	\item Variance of $X$
	\begin{equation}
		\text{Var}(X) = \sigma^{2} = \sum \frac{(X_{i} - \bar{X})^{2}}{n - 1}
		\label{eq:variance-random-variable}
	\end{equation}
	\item Standard Deviation
	\begin{itemize}
		\item of $X$
		\begin{equation}
			SD(X) = \sqrt{\sigma^{2}} = \sigma
			\label{eq:random-variable-standard-deviation}
		\end{equation}
		\item of $\bar{X}$
		\begin{equation}
			SD(\bar{X}) = \frac{\sigma}{\sqrt{n}}
			\label{eq:standard-error-of-the-mean}
		\end{equation}
		where $n$ is the sample size.
		Also known as the \emph{standard error of the mean}.
	\end{itemize}
\end{itemize}

\subsection{Sampling from a Normal Distribution}\label{subsec:sampling-from-a-normal-distribution}
\begin{itemize}
	\item For any sample size $n$, the sampling distribution of $\bar{X}$ is \emph{normal} if the population $X$ from which the sample is drawn is normally distributed.
	\item If $X$ is normal, then we can transform it into the \emph{standard normal random variable} as:
	\begin{itemize}
		\item For a sampling distribution:
		\begin{equation}
			\begin{aligned}
				Z &= \frac{\bar{X} - \hyperref[eq:expected-value-random-variable]{E(\bar{X})}}{\hyperref[eq:standard-error-of-the-mean]{\text{SD}(\bar{X})}}\\
				&= \frac{\bar{X} - \mu}{\frac{\sigma}{\sqrt{n}}}
			\end{aligned}
			\label{eq:normal-sampling-sample-distribution}
		\end{equation}
		\item For a distribution of the values of $X$.
		\begin{equation}
			\begin{aligned}
				Z &= \frac{X - E(X)}{\text{SD}(X)}\\
				&= \frac{X - \mu}{\sigma}
			\end{aligned}
			\label{eq:normal-sampling-distribution}
		\end{equation}
	\end{itemize}
\end{itemize}

\subsection{The Central Limit Theorem}\label{subsec:the-central-limit-theorem}
\lo{Explain the importance of the central limit theorem.}
\begin{itemize}
	\item For any population $X$ \w\ expected value $\mu$ and standard deviation $\sigma$, the sampling distribution of $\bar{X}$ will be approximately normal if the sample size $n$ is sufficiently large.
	\item As a general guideline, the normal distribution approximation is justified when $n \ge 30$.
	\item As before, if $\bar{X}$ is approximately normal, then we can transform it using~\eqref{eq:normal-sampling-sample-distribution}.
\end{itemize}

\section[Sampling Distribution of the Sample Proportion]{The Sampling Distribution of the Sample Proportion}\label{sec:the-sampling-distribution-of-the-sample-proportion}
\lo{Describe the properties of the sample distribution of the sample proportion.}
\begin{itemize}
	\item \definition{Estimator}{Sample proportion $\bar{P}$ is used to estimate the population parameter $p$.}
	\item \definition{Estimate}{a particular value of the estimator $\bar{p}$.}
\end{itemize}

\subsection[EV and SD of the Sample Proportion]{The Expected Value and Standard Deviation of the Sample Proportion}\label{subsec:the-expected-value-and-standard-deviation-of-the-sample-proportion}
\begin{itemize}
	\item The expected value of $\bar{P}$ is
	\begin{equation}
		E(\bar{P}) = p
		\label{eq:ev-sample-proportion}
	\end{equation}
	\item The standard deviation of $\bar{P}$ is
	\begin{equation}
		\text{SD}(\bar{P}) = \sqrt{\frac{p(1-p)}{n}}
		\label{eq:sd-sample-proportion}
	\end{equation}
\end{itemize}

\subsection[C.L.T. for the Sample Proportion]{The Central Limit Theorem for the Sample Proportion}\label{subsec:the-central-limit-theorem-for-the-sample-proportion}
\begin{itemize}
	\item For any population proportion $p$, the sampling distribution of $\bar{P}$ is approximately normal if the sample size $n$ is sufficiently large.
	\item As a general guideline, the normal distribution approximation is justified when $np \geq 5$ and $n(1-p) \geq 5$.
	\item If $\bar{P}$ is normal, we can transform it into the standard normal random variable as
	\begin{equation}
		\begin{aligned}
			Z &= \frac{\bar{P} - \hyperref[eq:ev-sample-proportion]{E(\bar{P})}}{\hyperref[eq:sd-sample-proportion]{SD(\bar{P})}}\\
			&= \frac{\bar{P} - p}{\sqrt{\frac{p(1-p)}{n}}}
		\end{aligned}
		\label{eq:standard-normal-sample-proportion}
	\end{equation}
	\item Therefore, any value $\bar{p}$ on $\bar{P}$ has a corresponding value $z$ on $Z$ given by
	\begin{equation}
		z = \frac{\bar{p} - p}{\sqrt{\frac{p(1-p)}{n}}}\\
		\label{eq:standard-normal-sample-proportion-estimate}
	\end{equation}
\end{itemize}

\section{The Finite Population Correction Factor}\label{sec:the-finite-population-correction-factor}
\lo{Use a finite population correction factor.}
\begin{itemize}
	\item Used to reduce the sampling variation of $\bar{X}$.
	\item The resulting standard deviation is
	\begin{equation}
		SD(\bar{X}) = \frac{\sigma}{\sqrt{n}} \left( \sqrt{\frac{N-n}{N-1}} \right)
		\label{eq:sd-finite-population-correction-factor}
	\end{equation}
	\item The transformation of $\bar{x}$ to $Z$ is made accordingly.
	\item Apparently, only used when $\frac{n}{N} > 5\%$.
\end{itemize}

\subsection[Finite Population Correction Factor for the Sample Proportion]{The Finite Population Correction Factor for the Sample Proportion}\label{subsec:the-finite-population-correction-factor-for-the-sample-proportion}
\begin{itemize}
	\item Used to reduce the sampling variation of the sample proportion $\bar{P}$.
	\item The resulting standard deviation is:
	\begin{equation}
		SD(\bar{P}) = \sqrt{\frac{p(1-p)}{n}} \left( \sqrt{\frac{N-n}{N-1}} \right)
		\label{eq:sd-finite-population-correction-factor-proportion}
	\end{equation}
	\item The transformation of $\bar{P}$ to $Z$ is made accordingly.
\end{itemize}

\section{Statistical Quality Control}\label{sec:statistical-quality-control}
\lo{Construct and interpret control charts from quantitative and qualitative data.}
\begin{itemize}
	\item Involves statistical techniques used to develop and maintain a firm's ability to produce high-quality goods and services.
	\item Two Approaches for Statistical Quality Control
	\begin{itemize}
		\item \hyperref[subsec:acceptance-sampling]{Acceptance Sampling}
		\item \hyperref[subsec:detection-approach]{Detection Approach}
	\end{itemize}
\end{itemize}

\subsection{Acceptance Sampling}\label{subsec:acceptance-sampling}
\begin{itemize}
	\item Used at the completion of a production process or service.
	\item If a particular product does not conform to certain specifications, then it is either discarded or repaired.
	\item Disadvantages
	\begin{itemize}
		\item It is costly to discard or repair a product.
		\item The detection of all defective products is not guaranteed.
	\end{itemize}
\end{itemize}

\subsection{Detection Approach}\label{subsec:detection-approach}
\begin{itemize}
	\item Inspection occurs during the production process in order to detect any nonconformance to specifications.
	\item Goal is to determine whether the production process should be continued or adjusted before producing a large number of defects.
	\item Types of variation.
	\begin{itemize}
		\item \hyperref[subsubsec:chance-variation]{Chance variation}.
		\item \hyperref[subsubsec:assignable-variation]{Assignable variation}.
	\end{itemize}
\end{itemize}

\subsubsection{Chance Variation (Common Variation)}\label{subsubsec:chance-variation}
\begin{itemize}
	\item Caused by a number of randomly occurring events that are part of the production process.
	\item Not controllable by the individual worker or machine.
	\item Expected, so not a source of alarm as long as its magnitude is tolerable and the end product meets specifications.
\end{itemize}

\subsubsection{Assignable variation}\label{subsubsec:assignable-variation}
\begin{itemize}
	\item Caused by specific events or factors that can usually be identified and eliminated.
	\item Identified and corrected or removed.
\end{itemize}

\subsection{Control Charts}\label{subsec:control-charts}
\begin{itemize}
	\item Developed by Walter A. Shewhart.
	\item A plot of calculated statistics of the production process over time.
	\item Production process is ``in control'' if the calculated statistics fall in an expected range.
	\item Production process is ``out of control'' if calculated statistics reveal an undesirable trend.
	\begin{itemize}
		\item For quantitative data--$\bar{x}$ chart.
		\item For qualitative data--$\bar{p}$ chart.
	\end{itemize}
\end{itemize}

\subsubsection{Control Charts for Quantitative Data}\label{subsubsec:control-charts-for-quantitative-data}
\begin{itemize}
	\item Centerline--the mean when the process is under control.
	\item Upper control limit (UCL)--set at $+3\sigma$ from the mean.
	\begin{equation}
		\mu + 3\frac{\sigma}{\sqrt{n}}
		\label{eq:ucl-quantitative}
	\end{equation}
	\begin{itemize}
		\item Points falling above the upper control limit are considered to be \emph{out of control}.
	\end{itemize}
	\item Lower control limit (LCL)--set at $-3\sigma$ from the mean.
	\begin{equation}
		\mu - 3\frac{\sigma}{\sqrt{n}}
		\label{eq:lcl-quantitative}
	\end{equation}
	\begin{itemize}
		\item Points falling below the lower control limit are considered to be \emph{out of control}.
	\end{itemize}
	\item Process is in control--all points fall \win\ the control limits.
\end{itemize}

\subsubsection{Control Charts for Qualitative Data}\label{subsubsec:control-charts-for-qualitative-data}
\begin{itemize}
	\item $\bar{p}$ chart (fraction defective or percent defective chart).
	\item Tracks proportion of defects in a production process.
	\item Relies on central limit theorem for normal approximation for the sampling distribution of the sample proportion.
	\item Centerline--the mean when the process is under control.
	\item Upper control limit (UCL)--set at $+3\sigma$ from the mean.
	\begin{equation}
		p + 3\sqrt{\frac{p(1-p)}{n}}
		\label{eq:ucl-qualitative}
	\end{equation}
	\begin{itemize}
		\item Points falling above the upper control limit are considered to be \emph{out of control}.
	\end{itemize}
	\item Lower control limit (LCL)--set at $-3\sigma$ from the mean.
	\begin{equation}
		p - 3\sqrt{\frac{p(1-p)}{n}}
		\label{eq:lcl-qualitative}
	\end{equation}
	\begin{itemize}
		\item Points falling below the lower control limit are considered to be \emph{out of control}.
	\end{itemize}
	\item Process is out of control--some points fall above the UCL.
\end{itemize}

%</Chapter-7>

\end{document}