\documentclass[
	lecture={3},
	title={Numerical Descriptive Measures}
]{msf502notes}

\begin{document}

\setcounter{chapter}{2}
%<*Chapter-3>
\chapter{Numerical Descriptive Measures}\label{ch:numerical-descriptive-measures}
\section{Investment Decision}\label{sec:investment-decision}

\begin{table}[H]
	\centering
	\caption{Investment Decision}
	\label{tab:investment-decision}
	\begin{tabular}{|*{6}{c|}}
		\hline
		\textbf{Year} & \textbf{Metals} & \textbf{Income} & \textbf{Year} & \textbf{Metals} & \textbf{Income}\\
		\hline
		2000 & -7.34 & 4.07 & 2005 & 43.79 & 3.12\\
		\hline
		2001 & 18.33 & 6.52 & 2006 & 34.30 & 8.15\\
		\hline
		2002 & 33.35 & 9.38 & 2007 & 36.13 & 5.44\\
		\hline
		2003 & 59.45 & 18.62 & 2008 & -56.02 & -11.37\\
		\hline
		2004 & 8.09 & 9.44 & 2009 & 76.46 & 31.77\\
		\hline
	\end{tabular}
\end{table}


\begin{itemize}
	\item Rebecca would like to
	\begin{enumerate}
		\item Determine the typical return of the mutual funds.
		\item Evaluate the investment risk of the mutual funds.
	\end{enumerate}
\end{itemize}
%
\begin{itemize}
	\item As an investment counselor at a large bank, Rebecca Johnson was asked by an inexperienced investor to explain the differences between two top-performing mutual funds:
	\begin{itemize}
		\item Vanguard's Precious Metals and Mining fund (Metals)
		\item Fidelity's Strategic Income Fund (Income)
	\end{itemize}
	\item The investor has collected sample returns for these two funds for years 2000 through 2009.
	These data are presented in the next slide.
\end{itemize}

\section{Measures of Central Location}\label{sec:measures-of-central-location}
\subsection{Mean}\label{subsec:mean}
\begin{itemize}
	\item The arithmetic mean is a primary measure of central location.
	\item Sample mean $\bar{x}$
	\begin{equation}
		\bar{x} = \frac{1}{n} \sum x_{i}
		\label{eq:sample-mean}
	\end{equation}
	\item Population mean $\mu$
	\begin{equation}
		\mu = \frac{1}{N} \sum x_{i}
		\label{eq:population-mean}
	\end{equation}
\end{itemize}

Metals fund metal return:
\[ \frac{-7.34 + 18.33 + 33.35 + 59.45 + 8.09 + 43.79 + 34.30 + 36.13 - 56.02 + 76.46}{10} = \frac{246.54}{10} = 24.654\% \]

Income fund mean return:
\[ \frac{4.07 + 3.12 + 6.52 + 8.15 + 9.38 + 5.44 + 18.62 - 11.37 + 9.44 + 31.77}{10} = \frac{85.14}{10} = 8.514\% \]

\subsection{Median}\label{subsec:median}

\subsection{Mode}\label{subsec:mode}

\subsection{Percentiles and Box Plots}\label{subsec:percentiles-and-box-plots}
\begin{itemize}
	\item In general, the $p$th percentile divides a data set into two parts:
	\begin{itemize}
		\item Approximately $p$\% of the observations have values less than the $p$th percentile;
		\item Approximately $(100 - p)$\% of the observations have values greater than the $p$th percentile.
	\end{itemize}
	\item Calculating the $p$th percentile:
	\begin{itemize}
		\item First, arrange the data in ascending order.
		\item Locate the position, $L_{p}$, of the $p$th percentile $\dots$
	\end{itemize}
	\item Consider the sorted data from the introductory case.
	\begin{table}[H]
		\centering
		\label{tab:percentile-example-data}
		\begin{tabular}{|l|*{10}{c|}}
			\hline
			Position & 1 & 2 & 3 & 4 & 5 & 6 & 7 & 8 & 9 & 10\\
			\hline
			Value & 4.07 & 3.12 & 6.52 & 8.15 & 9.38 & 5.44 & 18.62 & -11.37 & 9.44 & 31.77\\
			\hline
		\end{tabular}
	\end{table}
	\item For the 25th percentile, we locate
\end{itemize}
%
\begin{itemize}
	\item A box plot allows you to:
	\begin{itemize}
		\item Graphically display the distribution of a data set.
		\item Compare two or more distributions.
		\item Identify outliers in a data set.
	\end{itemize}
	\item Detecting outliers
	\begin{itemize}
		\item Calculate IQR = $Q_{3} - Q_{1}$
		\item Calculate $1.5 \times IQR$
		\item There are outliers if:
		\begin{itemize}
			\item $Q_{1} - \min > 1.5\text{IQR}$, or if
			\item $\max - Q_{3} > 1.5\text{IQR}$, or if
		\end{itemize}
	\end{itemize}
\end{itemize}

\subsection{Geometric Mean}\label{subsec:geometric-mean}
For multiperiod returns $R_{1}, R_{2}, \dots, R_{n}$, the geometric mean return $G_{R}$ is calculated as:
\begin{equation}
	G_{R} = \sqrt[n]{\left( 1 + R_{1} \right)\left( 1 + R_{2} \right)\dots\left( 1 + R_{n} \right)} - 1
	\label{eq:geometric-mean}
\end{equation}
where $n$ is the number of multiperiod returns.

\section{Measures of Dispersion}\label{sec:measures-of-dispersion}
\begin{itemize}
	\item Measures of dispersion gauge the variability of a data set.
	\item Measures of dispersion include:
	\begin{itemize}
		\item Range
		\item Mean Absolute Deviation (MAD)
		\item Variance and Standard Deviation
		\begin{itemize}
			\item In finance, standard deviation of a return is known as volatility
		\end{itemize}
		\item Coefficient of Variation (CV)
	\end{itemize}
\end{itemize}

\subsection{Mean Absolute Deviation (MAD)}\label{subsec:mean-absolute-deviation-(mad)}
\begin{itemize}
	\item MAD is an average of the absolute difference of each observation from the mean.
\end{itemize}

Sample MAD:
\begin{equation}
	\text{Sample MAD} = \frac{\sum | x_{i} - \bar{x} |}{n}
	\label{eq:sample-mad}
\end{equation}

\begin{equation}
	\text{Population MAD} = \frac{\sum | x_{i} - \mu |}{N}
	\label{eq:population-mad}
\end{equation}

\subsection{Variance and Standard Deviation}\label{subsec:variance-and-standard-deviation}
For a given sample,
\begin{equation}
	s^{2} = \frac{\sum \left( x_{i} - \bar{x} \right)^{2} }{n-1} \text{ and } s = \sqrt{s^{2}}
	\label{eq:sample-variance}
\end{equation}
where $s$ is the sample standard deviation and $s^{2}$ is the sample variance.

For a given population:
\begin{equation}
	\sigma^{2} = \frac{\sum \left( x_{i} - \mu \right)^{2} }{N} \text{ and } \sigma = \sqrt{\sigma^{2}}
	\label{eq:population-variance}
\end{equation}
where $\sigma$ is the standard deviation and $\sigma^{2}$ is the variance.

\begin{table}[H]
	\centering
	\caption{Volatility Index (VIX)}
	\label{tab:vix-index}
	\begin{tabular}{l|l|l}
		\textbf{Period} & \textbf{Typical VIX Levels} & \textbf{What It Means}\\
		\hline
		Quiet Markets & $\sim 10--15$ & Low fear, high confidence in equity returns\\
		Normal Conditions & $\sim 15--25$ & Modest uncertainty, moderate stability\\
		Crisis Conditions & $>30$ & Elevated fear--uncertain markets\\
		Peak Crises & $>60--80$ & Extreme panic or sharp dislocations
	\end{tabular}
\end{table}

\subsection{Coefficient of Variation (CV)}\label{subsec:coefficient-of-variation-(cv)}
\begin{itemize}
	\item CV adjusts for differences in the magnitudes of the means.
	\item CV is unitless, allowing easy comparison of mean-adjusted dispersion across different data sets.
\end{itemize}

\begin{equation}
	\text{Sample CV} = \frac{s}{\bar{x}}
	\label{eq:sample-cv}
\end{equation}

\begin{equation}
	\text{Population CV} = \frac{\sigma}{\mu} % TODO: Check this equation
	\label{eq:population-cv}
\end{equation}

\section{Sharpe Ratio}\label{sec:sharpe-ratio}
\begin{itemize}
	\item Measures the extra reward per unit of risk.
	\item For an investment $I$, the Sharpe ratio is computed as
	\begin{equation}
		\text{Sharpe Ratio} = \frac{\bar{x}_{I} - \bar{R}_{f}}{s_{I}}
		\label{eq:sharpe-ratio}
	\end{equation}
	where $\bar{x}_{I}$ is the mean return for the investment, $\bar{R}_{f}$ is the mean return for a risk-free asset, and $s_{I}$ is the standard deviation for the investment.
\end{itemize}

\section{Chebyshev's Theorem and the Empirical Rule}\label{sec:chebyshev's-theorem-and-the-empirical-rule}
\begin{itemize}
	\item Chebyshev's Theorem -- For any data set, the proportion of observation that lie \win\ $k$ standard deviations from the mean is at least $1 - \frac{1}{k^{2}}$, where $k$ is any number greater than 1.
	\item Consider a large lecture class \w\ 280 students.
	The mean score on an exam is 74 \w\ a standard deviation of 8.
	At least how many students scored \win\ 85 and 90?
	\item With $k=2$, we have $1 - \frac{1}{2^{2}} = 0.75 \dots$
\end{itemize}

\subsection{The Empirical Rule}\label{subsec:the-empirical-rule}
\begin{itemize}
	\item Approximately 68\% of all observations fall in the interval $\bar{x} \pm s$.
	\item Approximately 95\% of all observations fall in the interval $\bar{x} \pm 2s$.
	\item Approximately 99.7\% of all observations fall in the interval $\bar{x} \pm 3s$.
\end{itemize}

	% Section 3.7, summarizing grouped data, is not going to show up ever in class (homeworks, exams, etc)
%</Chapter-3>

\end{document}
