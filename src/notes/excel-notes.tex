\documentclass[12pt]{report}

\usepackage[title={Useful Excel Functions}]{msf502}

\begin{document}
%<*Excel-Functions>
\newcommand{\excel}[4][]{%
	\texttt{%
		\if#1\empty%
		#2%
		\else%
			\autoref{#1}~\hyperref[#1]{#2}%
		\fi
	}(#3) -- #4\\%
}
\setlength{\parindent}{0in}

\excel[eq:sample-variance]{STDEV.S}{$x_{1}$, $x_{2}$, $\dots$}{Sample standard deviation of a population.}

\excel[eq:binomial-probability-distribution]{BINOM.DIST}{\# of successes in trials, \# of trials, probability of success}{returns the individual term binomial distribution probability.}

\excel[eq:possion]{POISSON.DIST}{$x$, mean, cumulative=TRUE}{}

\excel[eq:normal-distribution]{NORM.S.DIST}{$z$, cumulative=TRUE}{gets the probability given $z$.}

\excel[eq:normal-distribution]{NORM.S.INV}{probability}{returns the inverse of the standard normal cumulative distribution. Finds the $z$-value given a probability.}

\excel[eq:exponential-cumulative-distribution]{EXPON.DIST}{$x$, $\lambda$, cumulative=TRUE}{}

\excel[eq:lognormal-pdf]{LOGNORM.DIST}{$x$, $\mu$, $\sigma$, cumulative=TRUE}{Returns the lognormal distribution of $x$, where $\ln(x)$ is normally distributed \w\ parameters $\mu$ and $\sigma$.}

\excel[eq:t-distribution]{T.DIST}{$x$, $df$, cumulative=TRUE}{returns the probability for the (left-tailed) $t$-distribution.}

\excel{T.DIST.2T}{$x$, $df$}{returns the probability for the two-tailed $t$-distribution.}

\excel{T.INV.2T}{probability, $df$}{Returns the two-tailed inverse of the $t$-distribution. Gets the two-tailed $t$-value for a given probability.}

\excel[eq:f-statistic]{F.DIST.RT}{$X$, $df1$, $df2$}{Returns the (right-tailed) $F$-probability distribution.}

\excel[subsec:right-tail-values]{F.INV.RT}{$\alpha$, $df1$, $df2$}{Returns a critical value such that the area in the right tail of the distribution is $\alpha$ (probability).}

\excel[eq:f-left-tail]{F.DIST}{$X$, $df1$, $df2$, cumulative=TRUE}{Returns the (left-tailed) $F$-probability distribution. If cumulative is \texttt{TRUE}, returns the cumulative distribution function; if FALSE, it returns the probability density function.}

\excel[eq:f-left-tail]{F.INV}{$\alpha$, $df1$, $df2$}{Returns a critical value such that the area in the left tail of the distribution is $\alpha$ (probability).}

\excel[eq:sample-covariance]{COVARIANCE.S}{$\text{array}_{1}$, $\text{array}_{2}$}{Calculates the correlation coefficient between two arrays.}

%</Excel-Functions>
\end{document}